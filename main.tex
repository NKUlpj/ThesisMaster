% !TEX encoding = UTF-8 Unicode
% !TEX program = xelatex
% !BIB program = biber
% !TEX TS-program = xelatex
% !BIB TS-program = biber
%%
%%  本模板方式编译: XeLaTeX + biber
%%
%%  注意: 在改变编译方式前应先删除 *.toc 和 *.aux 文件
%%
\documentclass[12pt,openright]{ctexbook}

% 引入NKThesis包
\usepackage[emptydoublepage]{NKThesis}   % 中文
%\usepackage[emptydoublepage,English]{NKThesis} % 英文

% 其它包按需添加
% \usepackage{amsmath}
% \usepackage{cases}
% \usepackage{multirow}
\usepackage{setspace} % 设置参考文献行距时的setstrech

% 参考文献
\addbibresource{nkthesis.bib}
% 图片文件夹
\graphicspath{{image/}}
% 使用三级节标题,如1.2.3.4
\setcounter{secnumdepth}{4}

% \includeonly{
% 	./tex/abstract,
% 	./tex/introduction,
% 	./tex/relatedwork,
% 	./tex/method,
% 	./tex/discussion,
% 	./tex/summary,
% 	./tex/references,
% 	./tex/acknowledgements,
% 	./tex/appendices,
% 	./tex/resume
% }
\begin{document}

%%%%%%%%%%%%%%%%%%%%%%%%%%%%%%%%%%%%%%%%%%%%%%%%%%%%%%%%%%%%%%%%%%%%%%%%%%%%%%%%
%  设置基本信息
%  注意:  逗号`,'是项目分隔符. 如果某一项的值出现逗号, 应放在花括号内, 如 {,}
%%%%%%%%%%%%%%%%%%%%%%%%%%%%%%%%%%%%%%%%%%%%%%%%%%%%%%%%%%%%%%%%%%%%%%%%%%%%%%%%
\NKTsetup{
	% 封面设置
	论文题目(中文) = 我是爱南开的,
	副标题         = ,
	论文题目(英文) = I Love Nankai,
	论文作者       = 周恩来,
	学号           = 19190062,
	指导教师       = {张伯苓 \quad 教授},
	申请学位       = 硕士, % 请参考2020版规范的2.2节,学术学位应包含类别(×学)和级别(硕士、博士)两部分,专业学位应写名称(如工程硕士)
	培养单位       = 南开大学, % 请参考2020版规范的2.1节,应填写所在学院(所)的规范全称
	学科专业       = 文科,
	研究方向       = 国际政治,
	答辩委员会主席 = {张伯苓},
	评阅人 = {严范孙},
	中图分类号     = ,
	UDC            = ,
	学校代码       = 10055,
	论文完成时间   = 二〇一八年四月, % 请参考2020版规范的2.1节,应填写提交评审的时间
	% 保密设置
	密级           = 公开,	% 公开 | 限制 | 秘密 | 机密, 若为公开, 不填以下三项
	非公开论文编号 = ,
	保密期限       = ,
	审批表编号     = ,
	% 其他信息
	批准日期       = ,
	答辩日期       = ,
	论文类别       = 学历硕士, % 博士 | 学历硕士 | 专业学位硕士 | 同等学力硕士
	院/系/所       = 文学部,
	联系电话       = 1234567890,
	Email          = ZhouEnlai@nankai.edu.cn,
	通讯地址(邮编) = 300000,
	备注           = {}
}

%%%%%%%%%%%%%%%%%%%%%%%%%%%%
% 论文开始部分
%%%%%%%%%%%%%%%%%%%%%%%%%%%%
% 摘要
% !TeX root = ../main.tex
% -*- coding: utf-8 -*-


\begin{zhaiyao}

这里输入中文摘要。
\end{zhaiyao}




\begin{guanjianci}
毕业论文;模板
\end{guanjianci}



\begin{abstract}


This is the abstract.

\end{abstract}



\begin{keywords}
Thesis; template
\end{keywords} 
% 论文目录
\tableofcontents
% \listoffigures

%%%%%%%%%%%%%%%%%%%%%%%%%%%%
% 论文主体章节
%%%%%%%%%%%%%%%%%%%%%%%%%%%%
% !TeX root = ../main.tex
% -*- coding: utf-8 -*-
% !TeX root = ../main.tex
% -*- coding: utf-8 -*-

\chapter{绪论}
\label{chpt:introduction}


本模板参照南开大学学位论文写作规范编写,
我在这里插入了参考文献\cite{li2023hibrowser}
仅仅提供了论文的基本格式,包括章节标题和正文字体、字号等等的设置。

% !TeX root = ../main.tex
% -*- coding: utf-8 -*-


\chapter{NKThesis 相关说明} 
\label{chpt:relatedwork}

\section{系统要求}
lorem
% !TeX root = ../main.tex
% -*- coding: utf-8 -*-

\chapter{常用包}
\label{chpt:method}

\section{The Tikz 绘图Package}
lorem
% !TeX root = ../main.tex
% -*- coding: utf-8 -*-

\chapter{讨论}



\section{\TeX\ 简介}
lorem
% !TeX root = ../main.tex
% -*- coding: utf-8 -*-
\chapter{总结展望}

任何问题可在GitHub上发起issue 

\href{https://github.com/NewFuture/NKThesis/issues/new}{github.com/NewFuture/NKThesis/issues/} 


%%%%%%%%%%%%%%%%%%%%%%%%%%%%
% 论文其他信息
%%%%%%%%%%%%%%%%%%%%%%%%%%%%
% !TeX root = ../main.tex
% -*- coding: utf-8 -*-

{
\zihaowu
\setlength{\bibitemsep}{0pt}
\setstretch{1.3}
% 2020版标准4.8
% 参考文献:标题要求同各章标题。文字部分:宋体10.5磅(或五号),英文用Times New Roman字体10.5磅(或五号),固定值行距16磅,段前段后0磅
% TeXLive2021中xelatex编译,尝试直接设置行距固定值,似对printbibliography无效,因此使用了1.3 \approx 16 \div 12

\printbibliography[title=参考文献]

}

% !TeX root = ../main.tex
% -*- coding: utf-8 -*-

%\makeschapterhead{致谢}
\chapter*{致谢}
{
\fangsong
\setlength{\baselineskip}{16pt}
\setlength{\parskip}{0pt}
% 2020版标准4.8
% 附录:标题要求同各章标题。文字部分仿宋12磅(或小四),固定值行距16磅,段前段后0磅
% 实操中有看到未使用仿宋的附录未被图书馆指明格式问题,因此或可酌情选择不设置字体。

感谢您使用本模板。
}

% !TeX root = ../main.tex
% -*- coding: utf-8 -*-

% !TeX root = ../main.tex
% -*- coding: utf-8 -*-

% 2020版要求4.8
% 个人简历、在学期间发表的学术论文与研究成果:
% 标题要求同各章标题。文字部分:宋体10.5磅(或五号),英文用Times New Roman字体10.5磅(或五号),固定值行距16磅,段前段后0磅,发表学术论文书写格式同参考文献
% 这一部分要求似乎较宽松,大小和行距或许可以不改。

\chapter*{个人简历、在学期间发表的学术论文与研究成果}

\section*{\leftline{个人简历}}

{
\zihaowu
\setlength{\baselineskip}{16pt}
\setlength{\parskip}{0pt}

xxx,出生于yyyy年mm月dd日。
在20yy年毕业于xx大学XX专业并获得xx士学位。
于20xx年至今在南开大学就读xxx研究生。

}

\section*{\leftline{研究生期间发表论文:}}
% 学术论文研究成果按发表的时间顺序列出
% (已发表的列在前面,已接收待发表的放在后面)
% 格式方便阅读为主可参考百度学术Google学术

% 注:这个要求似乎比参考文献宽松,因此没改。若强求格式,可改为普通文本并参照上一段修改字号行距等格式。

\begin{itemize}
	\item 周恩来. 周恩来选集[M]. 人民出版社, 1980.
	\item 周恩来. 周恩来外交文选[M]. 中央文献出版社, 1990.
\end{itemize}




% 其他成果有可添加
% \section*{\leftline{研究生期间其它成果:}}
% % 研究成果可以是在学期间参加的研究项目、申请的专利或获奖等
% \begin{itemize}
% 	\item 
% \end{itemize}


\end{document}
